% Useful macros for paper writing

%%%%%%%%%%%%%%%%%%%%%%%%% PROPOSAL OR PAPER-SPECIFIC MACROS
\newcommand{\percs}{{\small\textsf{PERCS}}\xspace}
\newcommand{\hl}{Headlight\xspace}
\newcommand{\idat}{Headlight\xspace}

%%%%%%%%%%%%%%%%%%%%%%%%%% PAGINATION
% \linepenalty=10000
% \frenchspacing
\setlength{\textwidth}{6.5in}
\setlength{\oddsidemargin}{0in}
\setlength{\textheight}{9.0in}
\setlength{\topmargin}{0in}
\setlength{\headheight}{0in}
\setlength{\headsep}{0in}
\setlength{\parindent}{15pt}
\setlength{\parskip}{0.1\baselineskip}

%%%%%%%%%%%%%%%%%%%%%%%%%% USEFUL LATIN ABBREVS
\newcommand{\etc}{\emph{etc.}\xspace}
\newcommand{\ie}{\emph{i.e.,}\xspace}
\newcommand{\eg}{\emph{e.g.,}\xspace}
\newcommand{\etal}{\emph{et al.}\xspace}

%\newcommand{\reducefiguretopvmargin}{\vspace*{0ex}}
%\newcommand{\reducefigurebottomvmargin}{\vspace*{0.0ex}}

%%%%%%%%%%%%%%%%%%%%%%%% REFERENCES
\newcommand{\secref}[1]{Section~\ref{sec:#1}}
\newcommand{\equaref}[1]{Eq.~(\ref{eq:#1})}
\newcommand{\figref}[1]{Figure~\ref{fig:#1}}
\newcommand{\algref}[1]{Alg.~(\ref{alg:#1})}
\newcommand{\ruleref}[1]{\textsc{Rule}~\ref{rule:#1}}

%%%%%%%%%%%%%%%%%%%%%%% MATH
\newtheorem{rul}{Rule}
\newcommand{\script}[1]{{{\cal{#1}}}}

%\newtheorem{theorem}{Theorem}[section]
%\newtheorem{lemma}[theorem]{Lemma}
%\newtheorem{proposition}[theorem]{Proposition}
%\newtheorem{corollary}[theorem]{Corollary}

% define fonts
\newcommand{\vct}[1]{\boldsymbol{#1}} % vector
\newcommand{\mat}[1]{\boldsymbol{#1}} % matrix

%%%% Special math symbols
\newcommand{\field}[1]{\mathbb{#1}}
\newcommand{\R}{\field{R}} % real domain
%\newcommand{\C}{\field{C}} % complex domain
\newcommand{\F}{\field{F}} % functional domain
%\newcommand{\T}{^{\top}\!\!} % transpose
\newcommand{\T}{^{\textrm T}} % transpose

%%% define constant
\newcommand{\cst}[1]{\mathsf{#1}}

%% operator in linear algebra, functional analysis
\newcommand{\inner}[2]{#1\cdot #2}
\newcommand{\norm}[1]{\|#1\|}
\newcommand{\twonorm}[1]{\|#1\|_2^2}
% operator in functios, maps such as M: domain1 --> domain 2
\newcommand{\Map}[1]{\mathcal{#1}}

% operator in probability: expectation, covariance, 
\newcommand{\ProbOpr}[1]{\mathbb{#1}} 
% independence
\newcommand\independent{\protect\mathpalette{\protect\independenT}{\perp}}
\def\independenT#1#2{\mathrel{\rlap{$#1#2$}\mkern2mu{#1#2}}}
% conditional independence
\newcommand{\cind}[3]{{#1} \independent{#2}\,|\,#3}
% conditional expectation
\newcommand{\cndexp}[2]{\ProbOpr{E}\,[ #1\,|\,#2\,]}
% operator in optimization
\DeclareMathOperator{\argmax}{arg\,max}
\DeclareMathOperator{\argmin}{arg\,min}
\newcommand{\todo}[1]{{\color{red}#1}}

% environment
\newtheorem{thm}{Theorem}

\newtheorem{theorem}{Theorem}[section]
\newtheorem{corollary}[theorem]{Corollary}
\newtheorem{lemma}[theorem]{Lemma}
\newtheorem{observation}[theorem]{Observation}
\newtheorem{proposition}[theorem]{Proposition}
\newtheorem{definition}[theorem]{Definition}
\newtheorem{claim}[theorem]{Claim}
\newtheorem{fact}[theorem]{Fact}
\newtheorem{assumption}[theorem]{Assumption}
\newtheorem{warning}[theorem]{Warning}
\newtheorem{conjecture}[theorem]{Conjecture}

\def\pnorm#1#2{\left\| #2 \right\|_{#1}}
\def\norm#1{\left\| #1 \right\|}
\def\maxnorm#1{\| #1 \|_{\max} }
\def\fnorm#1{\left\| #1 \right\|_{F}}
\def\expec#1#2{\mbox{\rm E}_{#1}\left[ #2 \right]}
\def\worst#1{\mbox{W}\left[ #1 \right]}
\def\average#1#2{\mbox{AVG}_{#1}\left[ #2 \right]}
\def\sizeof#1{\left|#1  \right|}
\def\prob#1#2{\mbox{\rm Pr}_{#1}\left[ #2 \right]}
\def\setof#1{\left\{{\let\st\colon #1 }\right\}}
\def\Zdn{\mathbb{Z}^d_n}



\def\vs#1#2#3{#1_{#2},\dots,#1_{#3}}

\def\allones{\mathbf{1}}
\def\origin{\mathbf{0}}

\def\aa{\mathbf{a}} \def\bb{\mathbf{b}} \def\cc{\mathbf{c}} \def\ee{\mathbf{e}} \def\00{\mathbf{0}} \def\ggg{\mathbf{g}} \def\hh{\mathbf{h}}
\def\uu{\mathbf{u}} \def\vv{\mathbf{v}} \def\xx{\mathbf{x}} \def\yy{\mathbf{y}}
\def\AA{\mathbf{A}} \def\BB{\mathbf{B}} \def\LL{\mathbf{L}} \def\PP{\mathbf{P}} \def\UU{\mathbf{U}}
\def\pp{\mathbf{p}} \def\qq{\mathbf{q}} \def\rr{\mathbf{r}} \def\ss{\mathbf{s}}
\def\ttt{\mathbf{t}} \def\xx{\mathbf{x}} \def\yy{\mathbf{y}} \def\ZZ{\mathbb{Z}}
\def\zz{\mathbf{z}} \def\LL{\mathbf{L}} \def\RR{\mathbf{R}} \def\MM{\mathbf{L}}
\def\NN{\mathbf{R}} \def\calG{\mathcal{G}} \def\calA{\mathcal{C}_A} \def\calI{\mathcal{C}_I}
\def\calS{\mathcal{S}} \def\calP{\mathcal{P}} \def\calC{\mathcal{C}}
\def\size#1{Size[#1]}
\def\bbE{\mathbb{E}} \def\ZZ#1#2{\mathbb{Z}^{#1}_{#2}} \def\ww{\mathbf{w}}
\def\calF{\mathcal{F}}
\def\calH{\cal H}

\def\xxtil{\mathbf{\tilde{x}}}
\def\pleq{\preccurlyeq}
\def\pgeq{\succcurlyeq}
\def\pinv#1{{#1}^{\dagger}}

\def\symP{\mathbb{P}}
\def\Reals#1{\mathbb{R}^{#1}}
\def\orig#1{\bar{#1}}


\def\Wtil{\tilde{W}}
\def\Gtil{\tilde{G}}

\def\rank#1{\mbox{\rm rank}\left( #1 \right)}
\def\Integers#1{\mathbb{Z}^{#1}}

\newcommand{\eat}[1]{}

%%%%%%%%%%%%%%%%%%%%%%% FIGURES OF VARIOUS KINDS
%%%%%%%%%%%%%%%%%%%%%%% Assumption: all figures are in figs directory

% Single image: location, scaling factor, caption
% sample: \scaleImage{h}{0.47}{K-ei3}{Performance under external
% interference}{label}

\newcommand{\scaleImage}[5]{
\begin{figure}[#1]
\centering
\includegraphics[width=#2\textwidth]{figures/#3}
%\reducefigurecaptionmargin
\caption{#4\label{fig:#5}}
%\reducefigurebottomvmarginsmall
\end{figure}
}

% args: side (l/r), width, filename, caption, label
% sample: \wrapImage{l}{2in}{rr.eps}{Privacy User Study}{label}
\newcommand{\wrapImage}[5]{
\begin{wrapfigure}{#1}{#2}
\centering
\includegraphics[width=#2]{figures/#3}
%\reducefigurecaptionmargin
\caption{#4\label{fig:#5}}
%\reducefigurebottomvmarginsmall
\end{wrapfigure}
}

% Multiple subimages in across page: subimages, caption, label
% Sample:
% \fullColumnFigs { 
% \subImage{0.22}{K-accuracy11b}{802.11b}{label}
% \subImage{0.22}{K-accuracy11a}{802.11a}{label}
% } {LIR of link pairs which \POLICE determined to be
% \textit{interfering} (squares) and \textit{non-interfering}
% (diamonds)} {accuracy}

\newcommand{\fullColumnFigs}[3]
{
\begin{figure*}[!tb]
  \centering
  {#1}
%\reducefigurecaptionmargin
\caption{#2\label{fig:#3}}
%\reducefigurebottomvmargin
\end{figure*}
}

\newcommand{\scaleTable}[4]{
\begin{table}[#1]
\centering
\includegraphics[width=#2\textwidth]{figures/#3}
%\reducefigurecaptionmargin
\caption{#4\label{tbl:#3}}
%\reducefigurebottomvmarginsmall
\end{table}
}

% Multiple subimages in one column: subimages, caption, label
% Sample:
% \oneColumnFigs { 
% \subImage{0.22}{K-accuracy11b}{802.11b}
% \subImage{0.22}{K-accuracy11a}{802.11a} 
% } {LIR of link pairs which \POLICE determined to be
% \textit{interfering} (squares) and \textit{non-interfering}
% (diamonds)} {accuracy}

\newcommand{\oneColumnFigs}[3]
{
\begin{figure}[t]
  \centering
  {#1}
%\reducefigurecaptionmargin
\caption{#2\label{fig:#3}}
%\reducefigurebottomvmargin
\end{figure}
}

% Subimage used above: width percentage, filename, caption
\newcommand{\subImage}[4]{% width, filename1, caption1, label1
    \hspace*{-2.0ex}
    \subfigure[#3]
    {
      \includegraphics[width=#1\textwidth]{figures/#2}
      \label{fig:#4}
    }
}

\newcommand{\subImagePadded}[5]{% figure_width, hpadding, filename1, caption1, label1
    \subfigure[#4]
    {
      \hspace{#2\textwidth}
      \includegraphics[width=#1\textwidth]{figures/#3}
      \hspace{#2\textwidth}
      \label{fig:#5}
    }
}

\newcommand{\subImageWithNoLabel}[2]{% width, filename1, caption1, label1
      \includegraphics[width=#1\textwidth]{figures/#2}
}

%%%%%%%%%%%%%%%%%%%%%%%%%%%%%%%%% PDFLATEX HACK
\epstopdfsetup{suffix=}

%%% Local Variables: 
%%% mode: latex
%%% TeX-master: "paper"
%%% End: 



%%%%% Squeezing space before sections, subsections, and re-defining
%%%%% paragraphs
\makeatletter
\let\origsection\section
\let\origsubsection\subsection
\let\origparagraph\paragraph

\renewcommand\section{\@ifstar{\starsection}{\nostarsection}}
\renewcommand\subsection{\@ifstar{\starsubsection}{\nostarsubsection}}
\renewcommand\paragraph{\@ifstar{\starpara}{\nostarpara}}

%% Change these
\newcommand\sectionprelude{\vspace{0em}}
\newcommand\sectionpostlude{\vspace{0em}}
\newcommand\subsectionprelude{\vspace{0em}}
\newcommand\subsectionpostlude{\vspace{0em}}
\newcommand\paraspace{\vspace*{-0.5ex}}

\newcommand\nostarsection[1]{\sectionprelude\origsection{#1}\sectionpostlude}
\newcommand\starsection[1]{\sectionprelude\origsection*{#1}\sectionpostlude}

\newcommand\nostarsubsection[1]{\subsectionprelude\origsubsection{#1}\subsectionpostlude}
\newcommand\starssubection[1]{\subsectionprelude\origsubsection*{#1}\subsectionpostlude}

\newcommand\starpara[1]{\paraspace\noindent\origparagraph*{\textbf{#1}}}
\newcommand\nostarpara[1]{\paraspace\noindent\origparagraph*{\textbf{#1}}}

\providecommand\subparagraph[1]{\paraspace\noindent\origparagraph*{\textit{#1}}}

\makeatother

%%% Backref magic
\DefineBibliographyStrings{english}{%
  backrefpage = {Cited on page},
  backrefpages = {Cited on pages},
}
\renewbibmacro{pageref}{%
  \iflistundef{pageref}
    {\printtext[parens]{Not Cited}} 
    {%
     \printtext[parens]{\ifnumgreater{\value{pageref}}{1}   
       {\bibstring{backrefpages}} 
       {\bibstring{backrefpage}}
       \printlist [pageref][-\value{listtotal}]{pageref}}}}    

\DeclareListFormat{pageref}{%
     % == 2 references
    \ifthenelse{\value{liststop} < 3}{\ifthenelse{\value{listcount}<\value{liststop}}{\hyperpage{#1} and }{\hyperpage{#1}}} %
    { % > 2 references
        \ifthenelse{\value{listcount}<\value{liststop}}
          {\hyperpage{#1}\addcomma\addspace}
          {\ifnumequal{\value{listcount}}{\value{liststop}}
            {and \hyperpage{#1}}
            {}%
          }%
    }%  
}

